%%
%% This is file `sample-sigconf.tex',
%% generated with the docstrip utility.
%%
%% The original source files were:
%%
%% samples.dtx  (with options: `sigconf')
%% 
%% IMPORTANT NOTICE:
%% 
%% For the copyright see the source file.
%% 
%% Any modified versions of this file must be renamed
%% with new filenames distinct from sample-sigconf.tex.
%% 
%% For distribution of the original source see the terms
%% for copying and modification in the file samples.dtx.
%% 
%% This generated file may be distributed as long as the
%% original source files, as listed above, are part of the
%% same distribution. (The sources need not necessarily be
%% in the same archive or directory.)
%%
%% The first command in your LaTeX source must be the \documentclass command.
\documentclass[sigplan]{acmart}

\usepackage{code}
\usepackage{graphicx}
\usepackage{placeins}

%%
%% \BibTeX command to typeset BibTeX logo in the docs
\AtBeginDocument{%
  \providecommand\BibTeX{{%
    \normalfont B\kern-0.5em{\scshape i\kern-0.25em b}\kern-0.8em\TeX}}}

%% Rights management information.  This information is sent to you
%% when you complete the rights form.  These commands have SAMPLE
%% values in them; it is your responsibility as an author to replace
%% the commands and values with those provided to you when you
%% complete the rights form.
\setcopyright{acmcopyright}
\copyrightyear{2021}
\acmYear{2021}
\acmDOI{10.1145/1122445.1122456}

%% These commands are for a PROCEEDINGS abstract or paper.
\acmConference[Woodstock '18]{Woodstock '18: ACM Symposium on Neural
  Gaze Detection}{June 03--05, 2018}{Woodstock, NY}
\acmBooktitle{Woodstock '18: ACM Symposium on Neural Gaze Detection,
  June 03--05, 2018, Woodstock, NY}
\acmPrice{15.00}
\acmISBN{978-1-4503-XXXX-X/18/06}


%%
%% Submission ID.
%% Use this when submitting an article to a sponsored event. You'll
%% receive a unique submission ID from the organizers
%% of the event, and this ID should be used as the parameter to this command.
%%\acmSubmissionID{123-A56-BU3}

%%
%% The majority of ACM publications use numbered citations and
%% references.  The command \citestyle{authoryear} switches to the
%% "author year" style.
%%
%% If you are preparing content for an event
%% sponsored by ACM SIGGRAPH, you must use the "author year" style of
%% citations and references.
%% Uncommenting
%% the next command will enable that style.
%%\citestyle{acmauthoryear}

%%
%% end of the preamble, start of the body of the document source.
\begin{document}

%%
%% The "title" command has an optional parameter,
%% allowing the author to define a "short title" to be used in page headers.
\title{(Programs), Proofs and Refutations\\(and Tests and Mutants)}

%%
%% The "author" command and its associated commands are used to define
%% the authors and their affiliations.
%% Of note is the shared affiliation of the first two authors, and the
%% "authornote" and "authornotemark" commands
%% used to denote shared contribution to the research.
\author{Alex Groce}
\affiliation{\institution{Northern Arizona University}\country{United States}}


%%
%% By default, the full list of authors will be used in the page
%% headers. Often, this list is too long, and will overlap
%% other information printed in the page headers. This command allows
%% the author to define a more concise list
%% of authors' names for this purpose.
\renewcommand{\shortauthors}{Alex Groce}

%%
%% The abstract is a short summary of the work to be presented in the
%% article.
\begin{abstract}

\end{abstract}

\begin{CCSXML}
<ccs2012>
<concept>
<concept_id>10011007.10010940.10010992.10010998.10011001</concept_id>
<concept_desc>Software and its engineering~Dynamic analysis</concept_desc>
<concept_significance>500</concept_significance>
</concept>
<concept>
<concept_id>10011007.10011074.10011099.10011102.10011103</concept_id>
<concept_desc>Software and its engineering~Software testing and debugging</concept_desc>
<concept_significance>500</concept_significance>
</concept>
</ccs2012>
\end{CCSXML}

\ccsdesc[500]{Software and its engineering~Dynamic analysis}
\ccsdesc[500]{Software and its engineering~Software testing and debugging}

\keywords{software testing software verificaton, proof, counterexample, tests}



\begin{abstract}
This essay consists of an imaginary discussion among a group of
  students after a computer science class, that presents some problems of
  (and partial solutions to) fundamental issues of program
  correctness.
  \end{abstract}



\maketitle


\section{The Dialogue}

The dialogue takes place in a classroom.  The last class of the day
has finished, and the professor is packing up laptop, 
display adapter, and mouse, and even remembering to turn off the
projection screen, as a few students remain sitting near each other.
The lingering students have become interested in a {\bf PROBLEM} (or
perhaps simply caught up in an argument).

The students ``recorded'' below do not, for the most part, stand in for particular stances
on software correctness; indeed, many of the students shift their
positions around as the discussion ranges (which, in fact, is possibly
a more accurate encapsulation of the intellectual careers of most
thinkers on these topics, albeit ludicrously compressed in temporal
terms, than a monolithic point of view would be!)

\vspace{0.1in}

\emph{(let's listen in)}

\vspace{0.1in}

\noindent {\bf PUPIL SIGMA:}  It just seems trivial to me.  I think the code here is simple enough that you can simply inspect it and see that it does what it should do.  The idea of binary search is slightly tricky to understand the very first time you see it, but once you understand how it works, writing code to ``do it'' does not require a PhD.

\vspace{0.1in}

\noindent {\bf PUPIL BETA:}  I'm not claiming it takes a PhD, or saying a PhD would help!  You saw how Dr. Omega messed it up the first time on the whiteboard.

\vspace{0.1in}

\noindent{\bf PUPIL DELTA:} \emph{(sotto voce)} That's because
Dr. Omega is a bit of an absent-minded flake.

\vspace{0.1in}

\noindent{\bf PUPIL SIGMA:}  Okay, that's true.  But Alpha noticed it
was wrong and once she explained why, we all understood.  But I want
to say something stronger than just that we all know the code is
correct now.  Look at the code...

\vspace{0.1in}

\emph{(let's look at that code ourselves (Figure \ref{fig:first}))}

\vspace{0.1in}

\begin{figure*}
  {\scriptsize
  \begin{code}
int binsearch(int* a, int key, unsigned int size) \{
  int low = 0;
  int high = size - 1;
  
  while (low <= high) \{ {\bf ON THIS LINE THE ``='' HAS CLEARLY BEEN ADDED LATER}
    int mid = (low + high) / 2;
    int midVal = a[mid];

    if (midVal < key)
      low = mid + 1;
    else if (midVal > key)
      high = mid - 1;
    else \{
      return mid; // key found
    \}
  \}
  return -1;  // key not found.
\}
\end{code}
}
\caption{C Code on the Whiteboard at the End of Class}
\label{fig:first}
\end{figure*}


\noindent{\bf PUPIL SIGMA:}  Now that we fixed that one problem, and
we all understand the basic idea of binary search, there's no use for
anything more formal or complicated.  There's so little room for bugs
here that any possible problems would be revealed as soon as we used
the code.  Think about the professor's mistake.  As soon as we started
using the code, it'd fail to find something present in the array.

\noindent{\bf PUPIL DELTA:} Are you sure?  I wonder how often it
fails.  I'll grant that if the code goes wrong often enough, someone
would notice, but you seem to be suggesting that even without tests,
we'd notice very quickly.  But what if we almost always search for
things not in the array?  Or even if we search for things present most
of the time, isn't it only going to show up when the item is where low
and high meet, and might that not be really rare?  Especially if the
array is very big!  I'd want thorough tests, and even for this code
I'm not sure how to make them!

\noindent{\bf PUPIL GAMMA:} If you varied the size of the array, in
your tests, that should help.  At size 1, this bug shows up every time
the element is present!

\noindent{\bf PUPIL ALPHA:}  Oh that's nice.  I bet that's a good
idea, to test systems that can vary their size on really small
versions.  You can probably test really thoroughly at small sizes,
too.  If an array only has one element, I think you've fully tested it
if you just check the case where the element is the one you're
searching for and the element is not the one you're searching for!

\noindent{\bf PUPIL DELTA:}  Anyway, I think the real problem isn't
how to test binary search.  I mean, that's fine, and I'm sure there
could even be computer scientists who just think about silly trivial
thngs like how to write better tests that find more bugs.  But don't
you think there's something deeper going on here, something that's a
\emph{real} problem?

\noindent{\bf PUPIL GAMMA:}  You're just jealous I thought of the
one-element array thing.

\noindent{\bf PUPIL BETA:} No, that's not it.  All philosophy-CS
double majors are like this, all the time.  It's annoying. Anyway, let's hear about that ``real'' problem.

\noindent{\bf PUPIL DELTA:}  The real problem is, imagine that we have
a set of perfect, absolutely thorough tests for binary search, or
whatever we're testing, let's call it program $\mathcal{P}$.  Or,
better yet, we have a complete proof of correctness of $\mathcal{P}$.
Now, those tests or that proof are going to be \emph{with respect to a
  specification of what $\mathcal{P}$ should do.}  Let's call that
specification $\mathcal{S}$.  Ok, we've demonstrated to everyone's
satisfaction that $\mathcal{P}$ satisfies $\mathcal{S}$.   I claim
we're not much better off than in the case where we just trust Sigma's
intuition that $\mathcal{P}$ is ``obviously right.''

\noindent{\bf PUPIL BETA:} How so?

\noindent{\bf PUPIL DELTA:} Because, we've just shoved back the
problem.  We were going to trust $\mathcal{P}$, or maybe trust Sigma.
Now we're just changing that to trusting $\mathcal{S}$!  That seems as
bad as trusting Sigma!

\noindent{\bf PUPIL SIGMA:} Hey!  What's so bad about trusting me!

\noindent{\bf PUPIL DELTA:} Sorry.  But seriously, we've just changed
the thing we have to place arbitrary trust in.  Even if our proofs or
tests are good, how do we know $\mathcal{S}$ is good?

\noindent{\bf PUPIL ALPHA:} $\mathcal{S}$  might be a lot simpler than
$\mathcal{P}$.

\noindent{\bf PUPIL DELTA:} Ok, I can imagine that might
probabilistically give us some more \emph{confidence}.  I don't want
to turn this into the general problem of epistemology, but I think
there is a practical problem here.  In the case of binary search, I
think there is a small amount of simplification from 
$\mathcal{P}$ to $\mathcal{S}$.  And a small gain.  But both are
fairly trivial.  It really is probably almost as good to just trust
Sigma as to trust $\mathcal{S}$.  Sigma made a perfect score on the
midterm, after all!

\vspace{0.1in}

\emph{(Delta walks to the whiteboard and points to the code.)}

\vspace{0.1in}

\noindent{\bf PUPIL DELTA:}  And maybe trusting Sigma or $\mathcal{S}$
is fine in this case.  But if $\mathcal{P}$ is something complicated,
just the specification $\mathcal{S}$ is going to be extraordinarily
complex, much more complex than the $\mathcal{P}$ for something like
binary search.  If we don't trust $\mathcal{P}$ for binary search, why
on earth would we ever trust something as complicated as $\mathcal{S}$
for a real problem?  What's an operating system's $\mathcal{S}$?

\noindent{\bf PUPIL GAMMA:}  In the real world, I think this ends up
being a social process, really.  I mean, you're right that it's about
trust, but not trusting just Sigma.  How do people come up with the
specification for an operating system, and decide if it's any good?  A
group of experts, I'd guess, work on it.  They toss ideas back and
forth, they look for counterexamples to proposals about how the system
should work.  They talk to test engineers, developers, users.

\noindent{\bf PUPIL DELTA:} You know, that sounds like what really
happens.  Trusting Sigma alone is no good, but a bunch of Sigmas...  I
don't know if that's ideal, but it's not nearly as bad.  And in
practice, it's what we have to rely on.  Even mathematical ``proofs''
are just things enough mathematicians agree to call proofs.  It
happens that in math, unlike in philosophy, there's pretty frequent
agreement on whether a proof \emph{is} a proof, but I've read enough
history of the field to know there are also cases where it took a lot
of argument and discussion to settle on the definitions and right form
of proof.  I think Euler's simple theorem about regular polyhedra was
one, even.

\noindent{\bf PUPIL EPSILON:}  I'm not sure I like this being a matter
of trusting a social process at all!  In mathematics, there \emph{is}
some way for experts to check each other, and to be honest, the stakes
aren't as high as with a self-driving car or a mission to Mars.
Groups of experts make mistakes all the time, especially if they are
on the same ``team'' and subject to groupthink.  One person's dominant
personality should not drive a discussion of how to make a safe and
reliable computer system!

\noindent{\bf PUPIL DELTA:} Of course.  But what else is there?  After
all, deliberation by the body of people who are in a position to make
a decision, by a social process, is how we determine who runs the country, a jury
of peers is how we decide if a person should go to jail or not, and so
forth.  In these cases, the people are far \emph{less} expert.  And
more is usually at stake!  Science and math also, basically operate
that way.  That's peer review.

\noindent{\bf PUPIL EPSILON:}  Sure, but I think the issue is ``what
else is there?''  It seems to me that computers offer us a chance to
finally do better, in certain limited parts of life.  Take proofs.
Before computers, proofs were inevitably checked by other
mathematicians.  The idea of reducing the steps in proofs to such
simple ones that other people could check the proofs exactly was, I
think, tossed around, by Leibniz and company.  But it couldn't work,
because nobody has the patience or attention to detail.  But before
computers, nobody also had the patience or attention to detail to
produce perfect logarithm tables.  Babbage's dream in part was to
replace legions of inattentive country clergy computing logarithms
poorly with a machine pumping them out perfectly.  To an extent he
woudln't have though possible, we have that now, and not just for
arithmetic problems, but with symbolic math tools.  There are people
who work on automated theorem provers, and while those aren't
replacing mathematicians or checking proofs like Fermat's last theorem
yet, I think there are people who have the basics of automatic
\emph{proof checking.}

\noindent{\bf PUPIL DELTA:}   Fine, but that's \emph{not} relevant to
my problem.  That's not checking $\mathcal{S}$.  It's checking the
proof $\mathcal{P}$.  Hmm, we called the program $\mathcal{P}$.  The
proof $\mathcal{Q}$ for ``QED''.

\noindent{\bf PUPIL EPSILON:}  Yes, there's nothing for your problem
right now.  That's why we have so many buggy specifications, not just
buggy programs.  But I think we should dream big, like Leibniz.  Sure,
his methods were fantastical at the time, so were Babbage's.  We can
dream about how to address the $\mathcal{S}$ problem, too!


\noindent{\bf PUPIL BETA:} Frankly, I don't think this problem of knowing if
you've proved or tested the right specification is quite as hard as
you all seem to think. I grant that \emph{sometimes} it is, but I
think often we have a program like this one, where it's easy.  Not
because, as Sigma originally suggested, binary search is so trivial,
but because binary search is \emph{equivalent to something that really
  is trivial.}

\noindent{\bf PUPIL GAMMA:} What do you mean?

\noindent{\bf PUPIL BETA:} I mean that the right specification for
binary search is very simple:  binary search works just like linear
search, only faster.  Ok, maybe the ``faster'' part is not so easy,
but in either a proof or a test, we just need to compare the
``tricky'' binary search results to the result for linear search.
Linear search is so simple I defy anyone, even the worst student in
this class, to get it seriously wrong!

\noindent{\bf PUPIL EPSILON:}  That's a really good idea!

\noindent{\bf PUPIL GAMMA:}  Except it doesn't work.   What if you are
using binary search on an array with duplicates?  The result won't
always be the same as for linear search  then.

\noindent{\bf PUPIL BETA:} Oh, fine, we can just say that if both
return that the value is found, but they report different positions,
we'll check if the right value is present in both
positions.

\noindent{\bf PUPIL GAMMA:} But then you are only using the linear
search to check ``not found'' results, and I imagine you could just
set up tests to know whether they are using a value in the array or
not.  And for proofs, I think proving equivalence to linear search for
``not found'' might be harder than just proving the right answer is
provided, since it's \emph{not} actually equivalent to linear search!

\noindent{\bf PUPIL EPSILON:}  Ok, maybe Beta hasn't saved us much
work here, where it's not quite equivalent and ``fixing'' the mismatch
doesn't seem much easier than just figuring out exactly what binary
search should do.  But I bet this is a good idea for the kinds of
programs we were mentioning above, where a person understanding fully
what a program ought to do seems so hard it's almost impossible.
Think about either a really abstract file system that doesn't have
hardware problems or efficiency issues, or a compiler that doesn't do
any optimizations, and maybe produces very slow, but simple, binary
code.  I don't know about proofs, but for testing at least, comparing
complex versions that have to be fast and practical for real-world use
to much simpler implementations that you couldn't use in real life...
That seems like a very nice way to test some things.

\noindent{\bf PUPIL DELTA:}  Yes, yes, that sounds practical.  But it
doesn't really address the underlying problem at all.  It's a cheap
hack that's sometimes available. How do we know the ``simple''
file system or compiler's ``idea of what it should do'' is right?  How
do we trust any specification that's too complicated to fit inside a
person's head.  Or, really, like we said, to fit inside multiple heads
at once so a group of reliable people can all agree that they are all
thinking of the same thing, and that thing is the right thing.  You're
just pushing the problem back one step, in a small set of cases, and
the same issue really comes up for the simple verison, if the problem
at hand is at all complicated, like your examples.

\noindent{\bf PUPIL EPSILON:}   It's still a partial solution, like I
suggested. Sometimes this will make the $\mathcal{S}$-problem easier.
That's a step in the right direction, and I'd rather trust that a
simple non-optimizing C compiler is right than trust that {\tt gcc
  -O3} does the right thing!

\noindent{\bf PUPIL BETA:}  Alpha, you've been awful quiet, that's not
like you. Is something wrong?

\noindent{\bf PUPIL ALPHA:}  No, I'm just thinking about the code.
Which the rest of you seem to have abandoned.  For instance, shouldn't
that {\tt unsigned int size} really be a {\tt size\_t}?

\noindent{\bf PUPIL GAMMA:}  One thing that worries me is that we're
talking as if the program  $\mathcal{P}$ is just finished, once and
for all, and then we test it or prove it and when we're satisfied we
call it a day.  Maybe that's true for a very small program like binary
search in a library, or as a homework assignment (if you are crazy and
prove your homework assignments, or even bother testing them).  But in
the real world, isn't code modified and changed all the time?  I feel
like a program in reality is either thrown away quickly (in which case
who cares if it's right?) or lives to be changed, maybe even
eventually having no lines of code in common with the original program.

\noindent{\bf PUPIL DELTA:}  Like the ship of Theseus!

\noindent{\bf PUPIL BETA:} Show-off... 

\noindent{\bf PUPIL DELTA:}  Seriously, it seems to me this is a possible reason
tests are even better than proofs!  Or at least useful even when we have a proof.
Imagine we change the code for some reason.  If we have really good tests, it's
easy to see if we made a mistake: we just run those tests!  But your
proof isn't something you can run.   You have to look at the proof and
the changed code, and think about whether the change breaks the
proof.  Another chance for human mistakes!

\noindent{\bf PUPIL EPSILON:} There are automated proofs, like I said,
where a
computer produces the proof, or at least checks that it's correct.

\noindent{\bf PUPIL DELTA:} Sure, but I think the tools for generating
proofs without human assistance are not that great right now.  That
might change, but right now it's true.  And the checkers don't seem
that helpful here:  I bet when you change your program, the
proof-checker just always says ``your proof no longer works.''

\noindent{\bf PUPIL EPSILON:}   In general, I think you're right. But
there are some limited tools that are almost fully automatic I think.
The Turing Award in 2007 was given\footnote{to Clarke, Emerson, and Sifakis}for a technique called model
checking, that is fairly limited in application, but really does, sort of,
compute the proof automatically.  We talked about it in automata
theory class.

\noindent{\bf PUPIL DELTA:}  This does suggest a way to think
abstractly, though not practically, about how good a test, or proof...
or maybe even an $\mathcal{S}$ is?  We want to know \emph{how many bugs does
  it catch?}  Of course, that's sort of useless.  We don't know all
the bugs there might be, or I guess we'd just go down that list and
check for each of them.  The list of all bugs is just a kind of ``super-$\mathcal{S}$''.  But it can get us a little beyond just
$\mathcal{S}$, in that if we have a program, and we release it to
users, and they complain about some awful behavior, even if we failed
to put ``no doing that!'' in $\mathcal{S}$, we can easily agree it's
bad, and revise $\mathcal{S}$.  That bad behavior is in the ``set of
all bugs'' at least in theory.

\noindent{\bf PUPIL EPSILON:}  Yes, but I'm looking for practical
solutions, ways computers could help us out.

\noindent{\bf PUPIL DELTA:}  Sometimes thinking about the problem
definition in the most abstract terms can be practical, you know.
That's part of the lesson of philosophy.

\noindent{\bf PUPIL BETA:}  Oh brother.

\noindent{\bf PUPIL EPSILON:}  No, wait.  I think you have something
there.  Imagine a ``bugginator'' -- a computer program that takes
another computer program, and changes it to all possible buggy
versions.  Now, if we have a bugginator, the opposite of a
``debugger'' if a debugger did what it says on the tin, I guess, then
we could get somewhere.  We could just run the bugginator, and run all
our tests and check all our proofs.  Every bug that isn't detected
either shows a problem with our tests, or with our proofs, OR with
$\mathcal{S}$.  There would be some work in figuring out which one, of
course, maybe hard work.  But it'd be concrete and practical.  I mean,
this is basically what we do in Delta's scenario of customers
reporting problems or a Mars probe mysteriously crashing.  We do a
post-mortem and if the problem really is a software problem, we end up
changing the program, but if we're diligent engineers we obviously fix
up our tests, that's what regression testing is, and nobody can argue
that's some impractical thing that nobody in industry can use.

\noindent{\bf PUPIL GAMMA:}  One little problem, again.   We don't
have a bugginator.  How can you make a bugginator?

\noindent{\bf PUPIL EPSILON:}   Ok, you can't.  But perhaps you could
make a partial bugginator.  Inject \emph{some} bugs automatically.  I
guess the lousy version would be to hire people to inject bugs.  Or
maybe an LLM can come up with good bugs?

\noindent{\bf PUPIL DELTA:}  I don't know how you'd get a good sample,
and I don't trust LLMs not to just make up the kinds of bugs engineers
already think about.  You'd want something principled, that comes up
with bug scenarios nobody normal will ever think of, the way the
fuzzers we saw in the software security class
come up with program inputs nobody considers.

\noindent{\bf PUPIL EPSILON:}  That's it!  Those fuzzers like AFL are
called mutation-based fuzzers because their basic loop is taking some
input, and changing it a little bit to see what the program does with
the mutated input.  So we get bugs for the bugginator by making small
random changes to $\mathcal{P}$.  You could call them ``mutants'' to
distinguish them from the set of all bugs, these are a sample that's
biased to things that are ``very close'' to the program we're
interested in, but it seems reasonable that most programs are fairly
close to correct, so the making small changes should make the program
``almost'' correct but not quite.

\noindent{\bf PUPIL GAMMA:}   Not bad, not bad.  The bug that Dr. Omega
made is very close to the correct version of binary search.  So you'd
surely include that kind of thing among your ``mutants.''   You don't
even need anything smart here, you could just write some dumb thing
with regular expressions to change arithmetic operators, or
comparisons, or invert {\tt if} conditions.

\noindent{\bf PUPIL EPSILON:}  I bet just commenting out random lines
of code could get you somewhere.

\noindent{\bf PUPIL GAMMA:} Let's go over to my dorm room and write
one of these.  I'd like to see how good my tests and my $\mathcal{S}$
are for the operating systems project.

\vspace{0.1in}

\emph{(All but one of the students walk out of the classroom, Beta and Delta holding
  hands (they are dating), Epsilon and Gamma discussing whether to
  write their bugginator in Python or Rust, Sigma talking on the
  phone, asking another student about hitting the gym. All is quiet
  for five minutes.  Then...)}

\vspace{0.1in}

\noindent{\bf PUPIL ALPHA:}  Hey!  Where'd you all go??? I think the code is still
wrong.  What happens if the size of {\tt a} is so big that {\tt high}
plus {\tt low} overflows?  I don't think your bugginator would catch
that problem with the proof or the tests or the specification.  The
real problem is the specification and proof and tests all assume
something about how big an array we want, and I think a proof might
assume an {\tt int} is an actual integer, not a computer int.  This is a problem
of the imagination.  How do we get an imaginator?


\begin{figure}
  {\scriptsize
  \begin{code}
\#define MAX\_SIZE 10

int main () \{
  int a[MAX\_SIZE];
  unsigned int SIZE = nondet\_uint();
  \_\_CPROVER\_assume(SIZE > 0);
  \_\_CPROVER\_assume(SIZE <= MAX\_SIZE);
  int k = nondet\_int();
  int present = 0;
  for (int i = 0; i < SIZE; i++) \{
    a[i] = nondet\_int();
    if (i > 0) \{
      \_\_CPROVER\_assume(a[i] >= a[i-1]);
    \}
    if (a[i] == k) \{
      present = 1;
    \}
  \}

  int r = binsearch(a, k, SIZE);
  if (r != -1) \{
    assert(a[r] == k);
  \} else \{
    assert(!present);
  \}
  \}
\end{code}
}
\caption{CBMC Proof Harness for Binary Search}
\label{fig:cbmc}
\end{figure}


\begin{figure}
{\scriptsize
  \begin{code}
\#include <algorithm>
\#include <deepstate/DeepState.hpp>

using namespace deepstate;

\#define MAX\_SIZE 32

TEST(Run, Bentley) \{
  int a[MAX\_SIZE];
  unsigned int SIZE = DeepState\_UIntInRange(1, MAX\_SIZE);
  int k = DeepState\_Int();
  int present = 0;

  for (int i = 0; i < SIZE; i++) \{
    a[i] = DeepState\_Int();
    if (a[i] == k) \{
      present = 1;
    \}
  \}
  std::sort(std::begin(a), \&a[SIZE]);
  if (!present \&\& DeepState\_Bool()) \{
    k = a[DeepState\_UIntInRange(0, SIZE-1)];
    present = 1;
  \}

  int r = binsearch(a, k, SIZE);
  if (r != -1) \{
    assert(a[r] == k);
  \} else \{
    assert(!present);
  \}
\}
\end{code}
}
\caption{DeepState Test Harness for Binary Search}
\label{fig:deepstate}
\end{figure}

\section{Postscript}

This dialogue is a tribute to, and reflection on, Imre Lakatos'
classic book-length dialogue
 \emph{Proofs and Refutation}~\cite{lakatos1963proofs}.  It also
owes a debt to MacKenzie's more recent sociological exploration of the
same issues, \emph{Mechanizing
  Proof: Computing, Risk, and Trust}~\cite{mackenzie2004mechanizing}.
To some extent the approach to thinking about proofs, tests, and their
meaning that (some of) the students arrive at is based on a Popperian~\cite{Popper} falsification methodology proposed in
work by my colleagues and
myself~\cite{groce2015verified,groce2018verified}.  All of these works
concern efforts to \emph{refute} the completeness or correctness of an
intellectual object, be it a mathematical definition, a mathematical
proof, a computer program, or a specification of a computer program.
The suggestive near-equivalences between the first two classes and the second two
classes are instructive, as is the similarity of both to the process of
scientific hypothesis and potentially falsifying experiment so central
to Popper's work.
Finally, the initial discussion owes a substantial debt to De Millo, Lipton, and
Perlis' (in)famous ``Social processes and proofs of theorems and
programs''~\cite{de1979social}.

A somewhat different (but still related) approach, less explicitly
presented in the dialogue, is that presented in, e.g., Turner's
\emph{Computational Artifacts}~\cite{turner2018computational}, where
programs are seen as \emph{machines} with a human-intended
\emph{purpose}.  It is interesting to consider that while metaphors
from this viewpoint are common (e.g., ``factory methods'') in the
field, the literature of program correctness is much more atuned to
programs as mathematical than as industrial artifacts (or, more
generally, as teleological products, like the wheel or the axe, of \emph{homo faber}).

The general issue of trust focusing on specifications, rather than
programs, may well loom larger in the future, if LLM or other AI-based
methods bring ``program synthesis'' (and, especially, potentially
hallucinatory and certainly stochastic, program synthesis) broadly
conceived, into wider usage.  One promising point is that while LLMs
can produce programs, which are hard to check, they can also produce
tests~\cite{gpttestgen,siddiq2023empirica}, which as the dialogue
suggests, may simplify the problem of trust.

Finally, the author was surprised to learn that Petricek, in 2017,
produced a very interesting dialogue, also featuring pupils Beta, Alpha, and so forth, in
similar homage to Lakatos, and covering some of the same topics as
this essay~\cite{DBLP:journals/programming/000117}!  Reading both
essays shows how much divergence is possible even given what amounts
to the same starting point, and the same unusual approach to similar
topics, and, for that matter, a probably substantial agreement by the
authors on certain core issues.

\subsection{Dijkstra's Objection}

To speak of these topics, in this way is, possibly, to infuriate the
ghost of Edsger Dijkstra.  In a 1977 position paper~\cite{dijkstra1977position}, Dijkstra argued
that discussing (other than, I assume, in formal terms of vacuity and
contradiction) $\mathcal{S}$ was, essentially, beyond the scope of the
\emph{scientific enterprise}:  best left to the general public, the
user (who may be sometimes ``written down as a fool''), or other
unspecified decisionmakers.  Dijkstra's concluding words are a
forthright challenge to this essay:

\begin{quote}
  As furthermore no scientific fruits are to be expected from dealing with fundamentally non-scientific issues, the scientist is justified in experiencing dealing with the non-scientific issue not only as a neglect of duty, but even as a waste of time.

P.S. The reader is mistaken if he thinks that he can send me a copy of Imre Lakatos’s \emph{Proofs and Refutations} for my education.
\end{quote}

Two responses are possible.  First, there is what I consider the
``coward's response:''  this is an essay, not a research paper, and
the discussion, as with the students, is not meant to contribute to
science; the author is acting as a member of the general public, or perhaps a fool;
certainly as an individual (like the students, as early in their
careers as it may be for them to have accomplished much along those
lines) who has sometimes been reasonably written down for a fool for
conceiving a poor specification.  I think this is a reasonable
argument, but it argues
too little.

That's because Dijkstra was simply wrong, I think, that there is no scientific or
mathematical way to approach the problem of the general suitabiltiy of
$\mathcal{S}$. it is certainly a hard problem, and one where purely mathematical
techniques are very likely \emph{limited} in effectiveness, but it is not outside the
scope of (computer) science, because it is possible to automate some of the
process of refutation and produce meaningful \emph{quantitative} measures of how much
$\mathcal{S}$ ``leaves out'' in terms of behaviors not prohibited.
The bugginator is reality.


I add that in a sense my personal sympathies do lie with
Dijkstra.  My own research interests have focused on non-social methods
for finding faults in software systems, those involving the execution
of complex computer programs, not the modification of development
processes, other than in the sense of incorporating the use of such tools.
The simple fact is, however, that such tools can help identify cases
where the specification itself should be questioned, and offer
opportunities to refine and revise $\mathcal{S}$ that do not arise
from a social process, but from a computation.

Furthermore, while Dijkstra might object to ``bugginator'' uses that
rely on testing, of which Dijkstra was notoriously skeptical, it is
possible to make use of mutants in fully formal proof-based settings.  This would
normally only be done in the context of automated proofs, for
practical reasons, but clarifies where I think Dijkstra would object
most strongly to the approach this essay endorses.  Of testing, he said that it can only show the presence
of bugs, not their absence; of mutation he would, I suspect, say that
it can only show the presence of (potential) bugs not captured by a
specification, not their absence.  In that, he is correct.

\section{Further Notes on Code and Tools / Code to Accompany the Dialogue}

The specific version of binary search that starts things rolling comes from Joshua Bloch's blog post~\cite{bloch} reporting
the bug the Alpha brilliantly detects.  It thus comes, fundamentally, from Jon Bentley's original version ``proven
correct'' in \emph{Programming Pearls}~\cite{Pearls}; proving binary
search has a long pedigree in the literature, e.g., back to
Hoare~\cite{hoare1971proof}.  Bloch's version  has
been changed from Java to C code.  The programs in question, and instructions
for proving and testing (and mutating~\cite{MutationSurvey}) them using CBMC~\cite{CBMCp} and
DeepState~\cite{goodman2018deepstate} (and UniversalMutator~\cite{SyntaxUM}), respectively,, can be found at
\url{https://github.com/agroce/onward24code}.  Some brief notes on the
connections between this code and the imaginary classroom discussion follow.

While linear search is a poor specification for binary search,
differential testing is indeed a powerful tool, widely used in compiler, file
system~\cite{ICSEDiff}, and container library testing, among other places.  Introduced
into the literature by McKeeman's classic paper focused on
compilers~\cite{Differential}, it has likely been independently
discovered many times in the history of testing.

Similarly, the ``just check 1'' approach discussed briefly relates to
ideas of using informal ``small model properties'' and, more
generally, bounded exhaustive testing~\cite{sullivan2004software},
as well as other heuristics for effective bug-finding.

\subsection{CBMC}


CBMC serves as an exemplar for \emph{proof}.  CBMC
translates C programs into goto-programs, and, eventually, into SAT or
SMT constraints, such that a satisfying assignment represents a
counterexample to the properties to be checked.  A proof of
unsatisfiability then is a proof of correctness for the program.

Strictly speaking, the
proof can be partial: CBMC is a \emph{bounded} model checker~\cite{BMC}, and so
requires the use of a bound on loop unrollings.  However, in the case
of binary search, a limit on unrollings of the search loop is part of
the full specification of correctness, so the proof is complete (since
CBMC can check that an execution exceeding provided loop bounds does
not exist).  The ``harness'' for CBMC is shown in
Figure~\ref{fig:cbmc}.  The code for binary search is omitted (see
Figure~\ref{fig:first}).


\subsection{DeepState}

DeepState exemplifies \emph{tests}.  While DeepState can make use of
symbolic execution, which can more resemble proof, if is primarily
used in conjunction with the more scalable approaches of random
testing and coverage-guided mutation-based fuzzing (e.g., with AFL~\cite{aflfuzz}).  The key idea is that while a
CBMC output of ``correct'' indicates that the program input satisfies
its specification (though not, as our students discuss, that the
specification itself is correct), DeepState may run for days without
finding a bug in an incorrect program.  On the other hand, for some
programs, DeepState will quickly find a bug, and CBMC will simply
exhaust the memory of the computer it is running on, and the patience
of the user, without accomplishing much of anything.  How often this
happens, vs. CBMC producing the bug quickly, if one exists, is hard to
know.

The ``harness'' for DeepState is shown in Figure~\ref{fig:deepstate}.
Again, the binary search code is omitted.  Note the larger size, due to the fact we
have to ``help out'' the testing rather than simply using {\tt ASSUME}
statements to force sorting and relying on exhaustiveness to check cases
where the item to be searched for is, in fact, present.

DeepState is one of many property-based testing~\cite{ClaessenH00,goldstein2024property} tools, unusual in that
it allows the use of state-of-the-art fuzzers for testing the
properties, and defines tests as parameterized/generalized unit
tests~\cite{ParamUnit}, with a syntax similar to that of Google's
GoogleTest framework~\cite{GoogleTest}.  Property-based testing is
likely to become far more popular in the future, and could be tightly integrated
with mutation testing, to make it easier to invent and
improve properties.

\subsection{UniversalMutator}

Finally, the idea of mutation testing is represented by UniversalMutator, which mutates code
in C and a number of other languages, using a more sophisticated
version of the approach proposed by Gamma~\cite{SyntaxUM}.  UniversalMutator, like other
mutation tools, acts as a very limited version of Epsilon's
hypothetical 
``bugginator.''  The imaginator remains up to us.


\bibliographystyle{plain}
\bibliography{bibliography}



\end{document}