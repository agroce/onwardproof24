%%
%% This is file `sample-sigconf.tex',
%% generated with the docstrip utility.
%%
%% The original source files were:
%%
%% samples.dtx  (with options: `sigconf')
%% 
%% IMPORTANT NOTICE:
%% 
%% For the copyright see the source file.
%% 
%% Any modified versions of this file must be renamed
%% with new filenames distinct from sample-sigconf.tex.
%% 
%% For distribution of the original source see the terms
%% for copying and modification in the file samples.dtx.
%% 
%% This generated file may be distributed as long as the
%% original source files, as listed above, are part of the
%% same distribution. (The sources need not necessarily be
%% in the same archive or directory.)
%%
%% The first command in your LaTeX source must be the \documentclass command.
\documentclass[sigplan]{acmart}

\usepackage{code}
\usepackage{graphicx}

%%
%% \BibTeX command to typeset BibTeX logo in the docs
\AtBeginDocument{%
  \providecommand\BibTeX{{%
    \normalfont B\kern-0.5em{\scshape i\kern-0.25em b}\kern-0.8em\TeX}}}

%% Rights management information.  This information is sent to you
%% when you complete the rights form.  These commands have SAMPLE
%% values in them; it is your responsibility as an author to replace
%% the commands and values with those provided to you when you
%% complete the rights form.
\setcopyright{acmcopyright}
\copyrightyear{2021}
\acmYear{2021}
\acmDOI{10.1145/1122445.1122456}

%% These commands are for a PROCEEDINGS abstract or paper.
\acmConference[Woodstock '18]{Woodstock '18: ACM Symposium on Neural
  Gaze Detection}{June 03--05, 2018}{Woodstock, NY}
\acmBooktitle{Woodstock '18: ACM Symposium on Neural Gaze Detection,
  June 03--05, 2018, Woodstock, NY}
\acmPrice{15.00}
\acmISBN{978-1-4503-XXXX-X/18/06}


%%
%% Submission ID.
%% Use this when submitting an article to a sponsored event. You'll
%% receive a unique submission ID from the organizers
%% of the event, and this ID should be used as the parameter to this command.
%%\acmSubmissionID{123-A56-BU3}

%%
%% The majority of ACM publications use numbered citations and
%% references.  The command \citestyle{authoryear} switches to the
%% "author year" style.
%%
%% If you are preparing content for an event
%% sponsored by ACM SIGGRAPH, you must use the "author year" style of
%% citations and references.
%% Uncommenting
%% the next command will enable that style.
%%\citestyle{acmauthoryear}

%%
%% end of the preamble, start of the body of the document source.
\begin{document}

%%
%% The "title" command has an optional parameter,
%% allowing the author to define a "short title" to be used in page headers.
\title{(Programs), Proofs and Refutations\\(and Tests and Mutants)}

%%
%% The "author" command and its associated commands are used to define
%% the authors and their affiliations.
%% Of note is the shared affiliation of the first two authors, and the
%% "authornote" and "authornotemark" commands
%% used to denote shared contribution to the research.
\author{Alex Groce}
\affiliation{\institution{Northern Arizona University}\country{United States}}


%%
%% By default, the full list of authors will be used in the page
%% headers. Often, this list is too long, and will overlap
%% other information printed in the page headers. This command allows
%% the author to define a more concise list
%% of authors' names for this purpose.
\renewcommand{\shortauthors}{Alex Groce}

%%
%% The abstract is a short summary of the work to be presented in the
%% article.
\begin{abstract}

\end{abstract}

\begin{CCSXML}
<ccs2012>
<concept>
<concept_id>10011007.10010940.10010992.10010998.10011001</concept_id>
<concept_desc>Software and its engineering~Dynamic analysis</concept_desc>
<concept_significance>500</concept_significance>
</concept>
<concept>
<concept_id>10011007.10011074.10011099.10011102.10011103</concept_id>
<concept_desc>Software and its engineering~Software testing and debugging</concept_desc>
<concept_significance>500</concept_significance>
</concept>
</ccs2012>
\end{CCSXML}

\ccsdesc[500]{Software and its engineering~Dynamic analysis}
\ccsdesc[500]{Software and its engineering~Software testing and debugging}

\keywords{software testing software verificaton, proof, counterexample, tests}


\maketitle

\section{The Dialogue}

The dialogue takes place in a classroom.  The last class of the day
has finished, and the professor is packing up laptop, 
display-adapter, and mouse, and even remembering to turn off the projection screen, as a few students remain sitting near each other.  The lingering students have become interested in a {\bf PROBLEM} (or perhaps simply caught up in an argument).

\vspace{0.1in}

\emph{(let's listen in)}

\vspace{0.1in}

\noindent {\bf PUPIL SIGMA:}  It just seems trivial to me.  I think the code here is simple enough that you can simply inspect it and see that it does what it should do.  The idea of binary search is slightly tricky to understand the very first time you see it, but once you understand how it works, writing code to ``do it'' does not require a PhD.

\vspace{0.1in}

\noindent {\bf PUPIL BETA:}  I'm not claiming it takes a PhD, or saying a PhD would help!  You saw how Dr. Omega messed it up the first time on the whiteboard.

\vspace{0.1in}

\noindent{\bf PUPIL DELTA:} \emph{(sotto voce)} That's because
Dr. Omega is a bit of an absentminded flake.

\vspace{0.1in}

\noindent{\bf PUPIL SIGMA:}  Okay, that's true.  But Alpha noticed it
was wrong and once she explained why, we all understood.  But I want
to say something stronger than just that we all know the code is
correct now.  Look at the code...

\vspace{0.1in}

\emph{(let's look at that code ourselves (Figure \ref{fig:first}))}

\vspace{0.1in}

\begin{figure*}
  \begin{code}
int binsearch(int* a, int key, int size) \{
  int low = 0;
  int high = size - 1;
  
  while (low <= high) \{ {\bf ON THIS LINE THE ``='' HAS CLEARLY BEEN ADDED LATER}
    int mid = (low + high) / 2;
    int midVal = a[mid];

    if (midVal < key)
      low = mid + 1;
    else if (midVal > key)
      high = mid - 1;
    else \{
      return mid; // key found
    \}
  \}
  return -1;  // key not found.
\}
\end{code}
\caption{C Code on the Whiteboard at the End of Class}
\label{fig:first}
\end{figure*}


\noindent{\bf PUPIL SIGMA:}  Now that we fixed that one problem, and
we all understand the basic idea of binary search, there's no use for
anything more formal or complicated.  There's so little room for bugs
here that any possible problems would be revealed as soon as we used
the code.  Think about the professor's mistake.  As soon as we started
using the code, it'd fail to find something present in the array.

\noindent{\bf PUPIL DELTA:} Are you sure?  I wonder how often it
fails.  I'll grant that if the code goes wrong often enough, someone
would notice, but you seem to be suggesting that even without tests,
we'd notice very quickly.  But what if we almost always search for
things not in the array?  Or even if we search for things present most
of the time, isn't it only going to show up when the item is where low
and high meet, and might that not be really rare?  Especially if the
array is very big!  I'd want thorough tests, and even for this code
I'm not sure how to make them!

\noindent{\bf PUPIL GAMMA:} If you varied the size of the array, in
your tests, that should help.  At size 1, this bug shows up every time
the element is present!

\noindent{\bf PUPIL ALPHA:}  Oh that's nice.  I bet that's a good
idea, to test systems that can vary their size on really small
versions.  You can probably test really thoroughly at small sizes,
too.  If an array only has one element, I think you've fully tested it
if you just check the case where the element is the one you're
searching for and the element is not the one you're searching for!

\noindent{\bf PUPIL DELTA:}  Anyway, I think the real problem isn't
how to test binary search.  I mean, that's fine, and I'm sure there
could even be computer scientists who just think about silly trivial
thngs like how to write better tests that find more bugs.  But don't
you think there's something deeper going on here, something that's a
\emph{real} problem?

\noindent{\bf PUPIL GAMMA:}  You're just jealous I thought of the
one-element array thing.

\noindent{\bf PUPIL BETA:} No, that's not it.  All philosophy-CS
double majors are like this, all the time.  It's annoying. Anyway, let's hear about that ``real'' problem.

\noindent{\bf PUPIL DELTA:}  The real problem is, imagine that we have
a set of perfect, absolutely thorough tests for binary search, or
whatever we're testing, let's call it program $\mathcal{P}$.  Or,
better yet, we have a complete proof of correctness of $\mathcal{P}$.
Now, those tests or that proof are going to be \emph{with respect to a
  specification of what $\mathcal{P}$ should do.}  Let's call that
specification $\mathcal{S}$.  Ok, we've demonstrated to everyone's
satisfaction that $\mathcal{P}$ satisfies $\mathcal{S}$.   I claim
we're not much better off than in the case where we just trust Sigma's
intuition that $\mathcal{P}$ is ``obviously right.''

\noindent{\bf PUPIL BETA:} How so?

\noindent{\bf PUPIL DELTA:} Because, we've just shoved back the
problem.  We were going to trust $\mathcal{P}$, or maybe trust Sigma.
Now we're just changing that to trusting $\mathcal{S}$!  That seems as
bad as trusting Sigma!

\noindent{\bf PUPIL SIGMA:} Hey!  What's so bad about trusting me!

\noindent{\bf PUPIL DELTA:} Sorry.  But seriously, we've just changed
the thing we have to place arbitrary trust in.  Even if our proofs or
tests are good, how do we know $\mathcal{S}$ is good?

\noindent{\bf PUPIL ALPHA:} $\mathcal{S}$  might be a lot simpler than
$\mathcal{P}$.

\noindent{\bf PUPIL DELTA:} Ok, I can imagine that might
probabilistically give us some more \emph{confidence}.  I don't want
to turn this into the general problem of epistemology, but I think
there is a practical problem here.  In the case of binary search, I
think there is a small amount of simplification from 
$\mathcal{P}$ to $\mathcal{S}$.  And a small gain.  But both are
fairly trivial.  It really is probably almost as good to just trust
Sigma as to trust $\mathcal{S}$.  Sigma made a perfect score on the
midterm, after all!

\vspace{0.1in}

\emph{(Delta walks to the whiteboard and points to the code.)}

\vspace{0.1in}

\noindent{\bf PUPIL DELTA:}  And maybe trusting Sigma or $\mathcal{S}$
is fine in this case.  But if $\mathcal{P}$ is something complicated,
just the specification $\mathcal{S}$ is going to be extraordinarily
complex, much more complex than the $\mathcal{P}$ for something like
binary search.  If we don't trust $\mathcal{P}$ for binary search, why
on earth would we ever trust something as complicated as $\mathcal{S}$
for a real problem?  What's an operating system's $\mathcal{S}$?

{\bf --------------------------------------}

\noindent{\bf PUPIL BETA:} I don't think this problem of knowing if
you've proved or tested the right specification is quite as hard as
you all seem to think. I grant that \emph{sometimes} it is, but I
think often we have a program like this one, where it's easy.  Not
because, as Sigma originally suggested, binary search is so trivial,
but because binary search is \emph{equivalent to something that really
  is trivial.}

\noindent{\bf PUPIL GAMMA:} What do you mean?

\noindent{\bf PUPIL BETA:} I mean that the right specification for
binary search is very simple:  binary search works just like linear
search, only faster.  Ok, maybe the ``faster'' part is not so easy,
but in either a proof or a test, we just need to compare the
``tricky'' binary search results to the result for linear search.
Linear search is so simple I defy anyone, even the worst student in
this class, to get it seriously wrong!

\noindent{\bf PUPIL EPSILON:}  That's a really good idea!

\noindent{\bf PUPIL GAMMA:}  Except it doesn't work.   What if you are
using binary search on an array with duplicates?  The result won't
always be the same as for linear search  then.

\noindent{\bf PUPIL BETA:} Oh, fine, we can just check that if both
return that the value is found,  the right value is present in both
positions.

\noindent{\bf PUPIL GAMMA:} But then you are only using the linear
search to check ``not found'' results, and I imagine you could just
set up tests to know whether they are using a value in the array or
not.  And for proofs, I think proving equivalence to linear search for
``not found'' might be harder than just proving the right answer is
provided, since it's \emph{not} actually equivalent to linear search!

\noindent{\bf PUPIL EPSILON:}  Ok, maybe Beta hasn't saved us much
work here, where it's not quite equivalent and ``fixing'' the mismatch
doesn't seem much easier than just figuring out exactly what binary
search should do.  But I bet this is a good idea for the kinds of
programs we were mentioning above, where a person understanding fully
what a program ought to do seems so hard it's almost impossible.
Think about either a really abstract file system that doesn't have
hardware problems or efficiency issues, or a compiler that doesn't do
any optimizations, and maybe produces very slow, but simple, binary
code.  I don't know about proofs, but for testing at least, comparing
complex versions that have to be fast and practical for real-world use
to much simpler implementations that you couldn't use in real life...
That seems like a very nice way to test some things.

\noindent{\bf PUPIL DELTA:}  Yes, yes, that sounds practical.  But it
doesn't really address the underlying problem at all.  It's a cheap
hack that's sometimes available.  But how do we know the ``simple''
file system or compiler's ``idea of what it should do'' is right?  How
do we trust any specification that's too complicated to fit inside a
person's head.  Or, really, like we said, to fit inside multiple heads
at once so a group of reliable people can all agree that they are all
thinking of the same thing, and that thing is the right thing.  You're
just pushing the problem back one step, in a small set of cases, and
the same issue really comes up for the simple verison, if the problem
at hand is at all complicated, like your examples.

{\bf --------------------------------------}

\noindent{\bf PUPIL GAMMA:}  One thing that worries me is that we're
talking as if the program  $\mathcal{P}$ is just finished, once and
for all, and then we test it or prove it and when we're satisfied we
call it a day.  Maybe that's true for a very small program like binary
search in a library, or as a homework assignment (if you are crazy and
prove your homework assignments, or even bother testing them).  But in
the real world, isn't code modified and changed all the time?  I feel
like a program in reality is either thrown away quickly (in which case
who cares if it's right?) or lives to be changed, maybe even
eventually having no lines of code in common with the original program.

\noindent{\bf PUPIL DELTA:}  Like the ship of Theseus!

\noindent{\bf PUPIL BETA:} Show-off... 

\noindent{\bf PUPIL DELTA:}  Seriously, it seems to me this is a possible reason
tests are even better than proofs!  Or at least useful even when we have a proof.
Imagine we change the code for some reason.  If we have really good tests, it's
easy to see if we made a mistake: we just run those tests!  But your
proof isn't something you can run.   You have to look at the proof and
the changed code, and think about whether the change breaks the
proof.  Another chance for human mistakes!

\noindent{\bf PUPIL EPSILON:} There are automated proofs, where a
computer produces the proof, or at least checks that it's correct.

\noindent{\bf PUPIL DELTA:} Sure, but I think the tools for generating
proofs without human assistance are not that great right now.  That
might change, but right now it's true.  And the checkers don't seem
that helpful here:  I bet when you change your program, the
proof-checker just always says ``your proof no longer works.''

\section{Postscript}

This (polyphonic) dialogue is a tribute to, and reflection on, Imre Lakatos' classic
work, \emph{Proof and Refutation}~\cite{lakatos1963proofs}.  It also
owes a debt to MacKenzie's more recent classic, \emph{Mechanizing
  Proof: Computing, Risk, and Trust}~\cite{mackenzie2004mechanizing}.
To some extent the approach to thinking about proofs, tests, and their
meaning that (some of) the students arrive at is based on a Popperian~\cite{Popper} falsification methodology proposed in
work by my colleagues and myself~\cite{groce2015verified,groce2018verified}.

The version of binary search that starts things rolling (and the bug
that unsettles the class at the end) comes from Joshua Bloch's blog post~\cite{bloch} reporting
the bug, and, thus, fundamentally, from Jon Bentley's original version ``proven
correct'' in \emph{Programming Pearls}~\cite{Pearls}.  The code has
been changed to C code.  The programs in question, and instructions
for proving and testing (and mutating) them using CBMC~\cite{CBMCp} and
DeepState~\cite{goodman2018deepstate} (and UniversalMutator~\cite{SyntaxUM}), respectively,, can be found at
\url{https://github.com/agroce/onward24code}.  Some brief notes on the
connection between this code and the above follows.

\subsection{CBMC}

CBMC serves as an exemplar for \emph{proof}.  CBMC
translates C programs into goto-programs, and, eventually, into SAT or
SMT constraints, such that a satisfying assignment represents a
counterexample to the properties to be checked.  A proof of
unsatisfiability then is a proof of correctness for the program.

Strictly speaking, the
proof can be partial: CBMC is a \emph{bounded} model checker, and so
requires the use of a bound on loop unrollings.  However, in the case
of binary search, a limit on unrollings of the search loop is part of
the full specification of correctness, so the proof is complete (since
CBMC can check that an execution exceeding provided loop bounds does
not exist).


\subsection{DeepState}

DeepState exemplifies \emph{tests}.  While DeepState can make use of
symbolic execution, which can more resemble proof, if is primarily
used in conjunction with the more scalable approaches of random
testing and coverage-guided fuzzing.  The key idea is that while a
CBMC output of ``correct'' indicates that the program input satisfies
its specification (though not, as our students discuss, that the
specification itself is correct), DeepState may run for days without
finding a bug in an incorrect program.  On the other hand, for some
programs, DeepState will quickly find a bug, and CBMC will simply
exhaust the memory of the computer it is running on, and the patience
of the user, without accomplishing much of anything.  How often this
happens, vs. producing the bug quickly, however, is hard to know.

\subsection{UniversalMutator}

Finally, the idea of mutation is represented by UniversalMutator, which mutates code
in C and a number of other languages.  UniversalMutator, like other
mutation tools, acts as a limited version of Epsilon's hypothetical
``bugginator.''


\bibliographystyle{plain}
\bibliography{bibliography}



\end{document}