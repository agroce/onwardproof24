%%
%% This is file `sample-sigconf.tex',
%% generated with the docstrip utility.
%%
%% The original source files were:
%%
%% samples.dtx  (with options: `sigconf')
%% 
%% IMPORTANT NOTICE:
%% 
%% For the copyright see the source file.
%% 
%% Any modified versions of this file must be renamed
%% with new filenames distinct from sample-sigconf.tex.
%% 
%% For distribution of the original source see the terms
%% for copying and modification in the file samples.dtx.
%% 
%% This generated file may be distributed as long as the
%% original source files, as listed above, are part of the
%% same distribution. (The sources need not necessarily be
%% in the same archive or directory.)
%%
%% The first command in your LaTeX source must be the \documentclass command.
\documentclass[sigplan]{acmart}

\usepackage{code}
\usepackage{graphicx}

%%
%% \BibTeX command to typeset BibTeX logo in the docs
\AtBeginDocument{%
  \providecommand\BibTeX{{%
    \normalfont B\kern-0.5em{\scshape i\kern-0.25em b}\kern-0.8em\TeX}}}

%% Rights management information.  This information is sent to you
%% when you complete the rights form.  These commands have SAMPLE
%% values in them; it is your responsibility as an author to replace
%% the commands and values with those provided to you when you
%% complete the rights form.
\setcopyright{acmcopyright}
\copyrightyear{2021}
\acmYear{2021}
\acmDOI{10.1145/1122445.1122456}

%% These commands are for a PROCEEDINGS abstract or paper.
\acmConference[Woodstock '18]{Woodstock '18: ACM Symposium on Neural
  Gaze Detection}{June 03--05, 2018}{Woodstock, NY}
\acmBooktitle{Woodstock '18: ACM Symposium on Neural Gaze Detection,
  June 03--05, 2018, Woodstock, NY}
\acmPrice{15.00}
\acmISBN{978-1-4503-XXXX-X/18/06}


%%
%% Submission ID.
%% Use this when submitting an article to a sponsored event. You'll
%% receive a unique submission ID from the organizers
%% of the event, and this ID should be used as the parameter to this command.
%%\acmSubmissionID{123-A56-BU3}

%%
%% The majority of ACM publications use numbered citations and
%% references.  The command \citestyle{authoryear} switches to the
%% "author year" style.
%%
%% If you are preparing content for an event
%% sponsored by ACM SIGGRAPH, you must use the "author year" style of
%% citations and references.
%% Uncommenting
%% the next command will enable that style.
%%\citestyle{acmauthoryear}

%%
%% end of the preamble, start of the body of the document source.
\begin{document}

%%
%% The "title" command has an optional parameter,
%% allowing the author to define a "short title" to be used in page headers.
\title{(Programs), Proofs and Refutations\\(and Tests and Mutants)}

%%
%% The "author" command and its associated commands are used to define
%% the authors and their affiliations.
%% Of note is the shared affiliation of the first two authors, and the
%% "authornote" and "authornotemark" commands
%% used to denote shared contribution to the research.
\author{Alex Groce}
\affiliation{\institution{Northern Arizona University}\country{United States}}


%%
%% By default, the full list of authors will be used in the page
%% headers. Often, this list is too long, and will overlap
%% other information printed in the page headers. This command allows
%% the author to define a more concise list
%% of authors' names for this purpose.
\renewcommand{\shortauthors}{Alex Groce}

%%
%% The abstract is a short summary of the work to be presented in the
%% article.
\begin{abstract}

\end{abstract}

\begin{CCSXML}
<ccs2012>
<concept>
<concept_id>10011007.10010940.10010992.10010998.10011001</concept_id>
<concept_desc>Software and its engineering~Dynamic analysis</concept_desc>
<concept_significance>500</concept_significance>
</concept>
<concept>
<concept_id>10011007.10011074.10011099.10011102.10011103</concept_id>
<concept_desc>Software and its engineering~Software testing and debugging</concept_desc>
<concept_significance>500</concept_significance>
</concept>
</ccs2012>
\end{CCSXML}

\ccsdesc[500]{Software and its engineering~Dynamic analysis}
\ccsdesc[500]{Software and its engineering~Software testing and debugging}

\keywords{software testing software verificaton, proof, counterexample, tests}


\maketitle

\section{The Dialogue}

The dialogue takes place in a classroom.  The last class of the day has finished, and the professor is packing up his laptop and turning off the projection screen, as a few students remain sitting near each other.  The lingering students have become interested in a {\bf PROBLEM} (or perhaps simply caught up in an argument).

\vspace{0.1in}

\emph{(let's listen in)}

\vspace{0.1in}

\noindent {\bf PUPIL SIGMA:}  It just seems trivial to me.  I think the code here is simple enough that you can simply inspect it and see that it does what it should do.  The idea of binary search is slightly tricky to understand the very first time you see it, but once you understand how it works, writing code to ``do it'' does not require a PhD.

\vspace{0.1in}

\noindent {\bf PUPIL BETA:}  I'm not claiming it takes a PhD, or saying a PhD would help!  You saw how Dr. Omega messed it up the first time on the whiteboard.

\vspace{0.1in}

\noindent{\bf PUPIL DELTA:} \emph{(sotto voce)} That's because
Dr. Omega is a bit of an absentminded flake.

\vspace{0.1in}

\noindent{\bf PUPIL SIGMA:}  Okay, that's true.  But Alpha noticed it
was wrong and once she explained why, we all understood.  But I want
to say something stronger than just that we all know the code is
correct now.  Look at the code...

\vspace{0.1in}

\emph{(let's look at that code ourselves (Figure \ref{fig:first})}

\begin{figure*}
  \begin{code}
int binsearch(int* a, int key, int size) \{
  int low = 0;
  int high = size - 1;
  
  while (low <= high) \{ {\bf ON THIS LINE THE ``='' HAS CLEARLY BEEN ADDED LATER}
    int mid = (low + high) / 2;
    int midVal = a[mid];

    if (midVal < key)
      low = mid + 1;
    else if (midVal > key)
      high = mid - 1;
    else \{
      return mid; // key found
    \}
  \}
  return -1;  // key not found.
\}
\end{code}
\caption{C Code on the Whiteboard at the End of Class}
\label{fig:first}
\end{figure*}


\noindent{\bf PUPIL SIGMA:}  Now that we fixed that one problem, and
we all understand the basic idea of binary search, there's no use for
anything more formal or complicated.  There's so little room for bugs
here that any possible problems would be revealed as soon as we used
the code.  Think about the professor's mistake.  As soon as we started
using the code, it'd fail to find something present in the array.

\noindent{\bf PUPIL DELTA:} Are you sure?  I wonder how often it
fails.  I'll grant that if the code goes wrong often enough, someone
would notice, but you seem to be suggesting that even without tests,
we'd notice very quickly.  But what if we almost always search for
things not in the array?  Or even if we search for things present most
of the time, isn't it only going to show up when the item is where low
and high meet, and might that not be really rare?  Especially if the
array is very big!  I'd want thorough tests, and even for this code
I'm not sure how to make them!


\noindent{\bf PUPIL DELTA:}  It seems to me this is another reason
tests are better, or at least useful even when we have a proof.
Imagine we change the code for some reason.  If we have tests, it's
easy to see if we made a mistake: we just run those tests!  But your
proof isn't something you can run.   You have to look at the proof and
the changed code, and think about whether the change breaks the
proof.  Another chance for human mistakes!

\section{Postscript}

This dialogue is a tribute to and reflection on Imre Lakatos' classic
work, \emph{Proof and Refutation}~\cite{lakatos1963proofs}.  It also
owes a debt to MacKenzie's more recent classic, \emph{Mechanizing
  Proof: Computing, Risk, and Trust}~\cite{mackenzie2004mechanizing}.
To some extent the approach to thinking about proofs, tests, and their
meaning is based on a Popperian falsification methodology proposed in
work by my colleagues and myself~\cite{groce2015verified,groce2018verified}.

The version of binary search that starts things rolling (and the bug
that startles the class) come from Joshua Bloch's blog post~\cite{bloch} reporting
the bug, and, thus, from Jon Bentley's original version ``proven
correct'' in \emph{Programming Pearls}~\cite{Pearls}.  The code has
been changed to C code, to fit the overall discussion (and use of
tools) better.

\bibliographystyle{plain}
\bibliography{bibliography}



\end{document}